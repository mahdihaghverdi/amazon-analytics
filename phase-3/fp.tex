\chapter{نقاط تابعی}
در این فصل به بررسی تقریبی نقاط تابعی پروژه که از ساختار‌های شکست کار استخراج می‌شوند  میپردازیم. نقاط تابعی نوشته شده به صورت تقریبی بدست آمده‌اند؛ چون نمیتوان از الان برای مثال،‌ برای یک 
\lr{RESTful API}
تعداد فایل‌های 
\lr{ILF}
و یا 
\lr{EIF}
را به صورت دقیق مشخص کرد.

\section{جدول نقاط تابعی}
این جدول به ترتیب از روی ساختار‌های شکست کار در فصل 
\ref{chap:wbs}
نوشته شده‌اند.

\begin{longtable}{|c|c|c|c|}
\caption{جدول نقاط تابعی}\\
\hline
ردیف &
نام &
نوع &
سطح \\
\hline
\tstep &
\lr{OpenAPI Specification} &
\lr{ILF} &
پیچیده \\
\hline
\hline
\tstep &
فایل‌های دایرکتوری 
\lr{shared}& 
\lr{ILF}&
متوسط \\
\hline
\tstep &
فایل 
\lr{models}& 
\lr{ILF}&
متوسط \\
\hline
\tstep &
فایل‌های 
\textit{\lr{repository pattern}}& 
\lr{ILF}&
سخت \\
\hline
\tstep &
فایل‌های 
\lr{services}&
\lr{ILF}&
متوسط \\
\hline
\tstep &
فایل‌های 
\lr{core}& 
\lr{ILF}&
سخت \\
\hline
\tstep &
فایل‌های 
\lr{API}&
\lr{ILF} &
متوسط \\
\hline
\tstep &
\lr{Dockerfile}& 
\lr{ILF}&
متوسط\\
\hline
\hline
\tstep &
فایل‌های سنسور‌های سخت‌افزاری& 
\lr{EIF}&
متوسط \\
\hline
\end{longtable}
\section{جدول نقاط ناسازگار}
