\documentclass[11pt, oneside]{book}

% URLs and hyperlinks ---------------------------------------
\usepackage{hyperref}
\hypersetup{
	colorlinks=true,
	linkcolor=blue,
	filecolor=magenta,      
	urlcolor=blue,
}
\usepackage{xurl}
%---------------------------------------------------

% titlepage -------------------------------------------------
\usepackage{pdfpages}
%------------------------------------------------------------

% commands --------------------------------------------
\newcommand{\amz}{\lr{Amazon} }
% -----------------------------------------------------------
\usepackage{xepersian}
\settextfont{Yas}
\setdigitfont{Yas}

\begin{document}
\frontmatter
    \begin{titlepage}
        \centering
        \includegraphics[width=4cm, height=4cm]{./images/logo}\par
        \vspace{2mm}            دانشگاه اصفهان \par
            دانشکده مهندسی کامپیوتر \par
        
        \vspace{1cm}
        {\huge \lr{\textbf{Amazon Analytics}}\par}
        \vspace{3cm}
        {\small\itshape                مهدی حق‌وردی\\
                سید محمدحسین هاشمی \par}
        
        \vfill \par	\vfill
        
        \vfill            استاد راهنما:‌ دکتر محمدرضا شعرباف \par
            دستیار استاد:‌ آقای رضا پورمحمدی
        \vfill
        
        % Bottom of the page
        {\large مهر ۱۴۰۲\par}
\end{titlepage}
\tableofcontents
\mainmatter

\chapter{معرفی}
پس از بررسی مواردی که برای پیاده‌سازی ارائه شده بود، موضوع \textit{منابع انسانی:‌ سیستم ارزیابی عملکرد کارمندان} انتخاب گردید.

به طور خلاصه این سیستم قرار است طبق چارت سازمانی معرفی شده در ادامه‌ی این سند، خدمات آماری به همراه جزئیات، به شرکت \amz و مدیران آن ارائه دهد. 

داده‌های مورد نیاز این خدمات آماری از آمار‌های ارائه شده توسط خود شرکت \amz و تمامی بازخورد (\lr{feedback})های مشتریان آن، جمع‌آوری و استفاده می‌شوند.

بر این اساس چون این سیستم قرار است به بررسی عملکرد کارمندان شرکت \amz بپردازد، و خدماتی که ارائه می‌دهد، خدمات آماری هستند، نام \lr{\textit{Amazon Analytics}} برای آن انتخاب گردید.

\chapter{مدل تجاری \lr{(Business Case)}}
\section{جدول \lr{RACI}}
\section{ماتریس تصمیم‌گیری راهبری}
\chapter{پیاده‌سازی نسخه‌ی دمو}
\chapter{قالب \lr{Brainstorming}}
\end{document}