\chapter{بخش ۵، ملاحظات پروژه\\\lr{(Project Considerations)}}

این فصل در مورد ریسک‌های پروژه 
\lr{Amazon Analytics}
بحث و بررسی مفصلی انجام می‌دهد.

لازم به ذکر است که ریسک‌های پروژه بر پایه‌ی ۳ ریسک اصلی نوشته شده‌اند:
\begin{enumerate}
\item 
بدست‌ آوردن داده،
\item 
اطمینان از دقیق و سالم بودن داده،
\item 
ارائه‌ی تحلیل دقیق.
\end{enumerate}

به علاوه فرض شده است که ریسک‌های مربوط به بودجه، تیم توسعه\RTLfootnote{برای مثال پیدا نکردن توسعه دهنده، رفتن یک توسعه‌دهنده و چنین ریسک‌های روتین} و چیز‌‌هایی از این قبیل، توسط خود شرکت آمازون فرض شده و استراتژی‌ها و پاسخ‌های مناسبی برای آنها در نظر گرفته شده، و در این فصل تنها به ریسک‌های مختص 
\lr{Amazon Analytics}
پرداخته شده است.

در بخش مفروضات، مفروضات لازم برای ریسک‌های ذکر شده در 
\ref{risks}،
گفته می‌شوند، اما ریسک‌هایی که خواهید خواند تماما بر اساس چارت سازمانی و تقسیم‌بندی انجام شده در سند مورد کاربرد تقسیم و نوشته شده‌اند.

\section{ریسک‌ها \lr{(Risks)}}\label{risks}

\subsection{ریسک‌های عملکردی}
\subsubsection{بخش \lr{Stock}}
\begin{itemize}
\item 
نداشتن اطلاعات دقیق از انبار‌های آمازون و پراکندگی کالاها
\item 
نداشتن اطلاعات (خصوصا محل سکونت) خریداران یک کالا
\end{itemize}

\subsubsection{بخش \lr{Site}}
ریسک‌های این قسمت تاثیر زیادی از بازخورد‌های دریافتی از کاربران آمازون گرفته‌اند.

\begin{itemize}
\item 
کاربران به محصولات و قسمت‌‌های مختلف هیچگونه بازخوردی نمی‌دهند
\item 
فید‌بک‌های کاربران واقعی نیستند (یا با ربات تولید شده‌اند یا دروغ‌اند)
\end{itemize}

\subsubsection{بخش \lr{Shipment}}
\begin{itemize}
\item

\item 

\item  

\end{itemize}

\subsection{ریسک‌های زیرساختی}
\subsubsection{بخش \lr{Data API}}
\begin{itemize}
\item

\item 

\item 
 
\end{itemize}

\subsubsection{بخش \lr{Process API}}
\begin{itemize}
\item

\item 

\item 
 
\end{itemize}

\subsubsection{بخش \lr{Web App}}
\begin{itemize}
\item

\item 

\item 
 
\item

\item 

\item  
\end{itemize}

\section{مشکلات \lr{(Issues)}}

\section{مفروضات \lr{(Assumptions)}}

\section{محدودیت‌ها \lr{(Constraints)}}

