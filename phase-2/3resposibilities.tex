\chapter{بخش ۳.۴، مسئولیت‌ها}

\section{پشتیبان‌ها}
\begin{itemize}
\item تعیین چشم انداز و اهداف سطح بالا برای پروژه
\item تصویب الزامات، جدول زمانی، منابع و بودجه
\item تصویب طرح پروژه و طرح کیفیت
\item اطمینان از شناسایی و مدیریت ریسک های عمده تجاری
\item تصویب هرگونه تغییر عمده در حوزه
\item دریافت صورتجلسه گروه بررسی پروژه و انجام اقدامات لازم
\item حل و فصل مسائل تشدید شده توسط مدیر پروژه / گروه بررسی پروژه
\item اطمینان از ایجاد ترتیبات پشتیبانی عملیاتی
\item ارائه پذیرش نهایی راه حل پس از اتمام پروژه
\end{itemize}

\section{گروه بازبینی}
\begin{itemize}
\item کمک به حامی پروژه در تعریف چشم انداز و اهداف پروژه
\item انجام بررسی های کیفیت قبل از اتمام هر نقطه عطف پروژه
\item حصول اطمینان از اینکه تمام ریسک های تجاری شناسایی و مدیریت می شوند
\item اطمینان از انطباق با استانداردها و فرآیندهای مشخص شده در طرح کیفیت
\item اطمینان از اینکه تمام اسناد قراردادی مناسب قبل از شروع پروژه وجود دارد
\end{itemize}

\section{مدیر}
\begin{itemize}
\item مستندسازی طرح تفصیلی پروژه و طرح کیفیت
\item اطمینان از تخصیص تمام منابع مورد نیاز به پروژه و تعیین تکلیف واضح
\item مدیریت منابع اختصاص یافته با توجه به محدوده تعریف شده پروژه
\item اجرای فرآیندهای پروژه زیر: زمان / هزینه / کیفیت / تغییر / ریسک / موضوع / تدارکات / ارتباطات / مدیریت پذیرش
\item نظارت و گزارش عملکرد پروژه (بازنگری: زمانبندی، هزینه، کیفیت و ریسک)
\item حصول اطمینان از انطباق با فرآیندها و استانداردهای مشخص شده در طرح کیفیت
\item گزارش و تشدید خطرات و مسائل پروژه
\item مدیریت وابستگی های متقابل پروژه
\item انجام تنظیمات لازم در پلان تفصیلی برای ارائه تصویری کامل از پیشرفت پروژه در هر زمان
\end{itemize}

\section{عضو تیم}
\begin{itemize}
\item انجام کلیه وظایف اختصاص داده شده توسط مدیر پروژه (طبق طرح پروژه)
\item گزارش پیشرفت اجرای وظایف به مدیر پروژه به صورت مکرر
\item حفظ کلیه اسناد مربوط به اجرای وظایف تخصیص یافته
\item افزایش خطرات و مسائلی که باید توسط مدیر پروژه حل شود
\end{itemize}